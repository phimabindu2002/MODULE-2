\documentclass[12pt,-letter paper]{article}
\usepackage{siunitx}
\usepackage{setspace}
\usepackage{gensymb}
\usepackage{xcolor}
\usepackage{caption}
%\usepackage{subcaption}
\doublespacing
\singlespacing
\usepackage[none]{hyphenat}
\usepackage{amssymb}
\usepackage{relsize}
\usepackage[cmex10]{amsmath}
\usepackage{mathtools}
\usepackage{amsmath}
\usepackage{commath}
%\usepackage{amsthm}https://www.overleaf.com/project/657d6a281358e1782949a240
\interdisplaylinepenalty=2500
%\savesymbol{iint}
\usepackage{txfonts}
%\restoresymbol{TXF}{iint}
\usepackage{wasysym}
\usepackage{amsthm}
\usepackage{mathrsfs}
\usepackage{txfonts}
\let\vec\mathbf{}
\usepackage{stfloats}
\usepackage{float}
\usepackage{cite}
\usepackage{cases}
\usepackage{subfig}
%\usepackage{xtab}
\usepackage{longtable}
\usepackage{multirow}
%\usepackage{algorithm}
\usepackage{amssymb}
%\usepackage{algpseudocode}
\usepackage{enumitem}
\usepackage{mathtools}
%\usepackage{eenrc}
%\usepackage[framemethod=tikz]{mdframed}
\usepackage{listings}
%\usepackage{listings}
\usepackage[latin1]{inputenc}
%%\usepackage{color}{   
%%\usepackage{lscape}
\usepackage{textcomp}
\usepackage{titling}
\usepackage{hyperref}
%\usepackage{fulbigskip}   
\usepackage{tikz}
\usepackage{graphicx}
\lstset{
  frame=single,
  breaklines=true
}
\let\vec\mathbf{}
\usepackage{enumitem}
\usepackage{graphicx}
\usepackage{siunitx}
\let\vec\mathbf{}
\usepackage{enumitem}
\usepackage{graphicx}
\usepackage{enumitem}
\usepackage{tfrupee}
\usepackage{amsmath}
\usepackage{amssymb}
\usepackage{mwe} % for blindtext and example-image-a in example
\usepackage{wrapfig}
\graphicspath{{figs/}}
\providecommand{\cbrak}[1]{\ensuremath{\left\{#1\right\}}}
\providecommand{\brak}[1]{\ensuremath{\left(#1\right)}}
%\providecommand{\norm}[1]{\left\lVert#1\right\rVert}
\newcommand{\myvec}[1]{\ensuremath{\begin{pmatrix}#1\end{pmatrix}}}
\newcommand{\augvec}[3]{\ensuremath{\begin{amatrix}{#1|#2}#3\end{amatrix}}}
\newcommand{\mydet}[1]{\ensuremath{\begin{vmatrix}#1\end{vmatrix}}}
\usepackage{subfig}\graphicspath{{/home/figures/fig}}

%\newcommand{\abs}[1]{\lvert#1\rvert}
%\newcommand{\norm}[1]{\lVert#1\rVert}
\providecommand{\sbrak}[1]{\ensuremath{{}\left[#1\right]}}
\providecommand{\brak}[1]{\ensuremath{\left(#1\right)}}
\providecommand{\cbrak}[1]{\ensuremath{\left\{#1\right\}}}
%\newcommand{\myvec}[1]{\ensuremath{\begin{pmatrix}#1%\end{pmatrix}}}
\newcommand{\myaugvec}[2]{\ensuremath{\begin{amatrix}{#1}#2\end{amatrix}}}
%\newcommand{\mydet}[1]{\ensuremath{\begin{vmatrix}#1%\end{vmatrix}}}
\usepackage{subfig}
%\graphicspath{/home/himabindu/figures}
\begin{document}
\title{\textbf{MATH-COMPUTING}}
\maketitle
\begin{enumerate}
 
    \item \textbf{Question(MATH-12.10.5.17):}
       Let $\vec{a}$ and $\vec{b}$ be two unit vectors and $\theta$ is the angle between them. Then $\vec{a}+\vec{b}$ is a unit vector.
    
 \begin{enumerate}[label=(\Alph*)]                     
 \item $\theta$=$\frac{\pi}{4}$
 \item $\theta$=$\frac{\pi}{3}$
  \item $\theta$=$\frac{\pi}{2}$
   \item $\theta$=$\frac{2\pi}{3}$
   \end{enumerate}
		 \textbf{solution:}

Given,
\begin{align}
	\norm{\vec{a}} &= \norm{\vec{b}}=1 
 \label{eq:eq1},\\
	\norm{\vec{a}+\vec{b}}&=1
 \label{eq:eq0}
 \end{align}
Squaring on both sides of \eqref{eq:eq0}, we get
\begin{align}
	\norm{\vec{a}+\vec{b}}^2 &=1^2
\\	
	\implies \norm{\vec{a}}^2 + \norm{\vec{b}}^2 + 2\vec{a}^{\top}\vec{b} &= 1
 \label{eq:eq2}
\end{align}

Substituting \eqref{eq:eq1} in \eqref{eq:eq2}, we get

\begin{align}
	\implies 1+1+2(\norm{\vec{a}}\norm{\vec{b}}\cos{\theta}) &=1\\
	\implies 2+2(\norm{\vec{a}}\norm{\vec{b}}\cos{\theta}) &=1\\
	\implies 2(\norm{\vec{a}}\norm{\vec{b}}\cos{\theta}) &=-1\\
	\implies (\norm{\vec{a}}\norm{\vec{b}}\cos{\theta}) &=\frac{-1}{2}
 \label{eq:eq3}
\end{align}
Substituting \eqref{eq:eq1} in \eqref{eq:eq3}, we get
\begin{align}
	\implies \cos{\theta} &=\frac{-1}{2}
	\\
	\implies \theta &=\frac{2\pi}{3}
\end{align}

Let, \begin{align}
	\vec{a} &= \myvec{\cos \theta_1 \\ \sin \theta_1}\\
	\vec{b} &= \myvec{\cos \theta_2 \\ \sin \theta_2}
\end{align}
Matrix multiplication of $\vec{a}.\vec{b}$ is:
\begin{align}
	\vec{a}^{\top}\vec {b}= \cos \brak{\theta_1 -\theta_2} &= \frac{-1}{2}\\
	\theta_1 - \theta_2& =\cos^{-1} \brak{\frac{-1}{2}}\\
	\theta_1 - \theta_2 &= \frac{2\pi}{3}\\
	\theta_1 &= \theta_2 +\frac{2\pi}{3}
\end{align}
  \begin{figure}[H]
	  \centering        
	  \includegraphics[width=\columnwidth]{/home/himabindu/figures/rer.png}
	  \caption{}       
	  \label{fig:python generated plot}
  \end{figure}
\end{enumerate}
\end{document}
