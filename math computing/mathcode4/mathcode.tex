  \documentclass[12pt,-letter paper]{article}
\usepackage{siunitx}
\usepackage{setspace}
\usepackage{gensymb}
\usepackage{xcolor}
\usepackage{caption}
%\usepackage{subcaption}
\doublespacing
\singlespacing
\usepackage[none]{hyphenat}
\usepackage{amssymb}
\usepackage{relsize}
\usepackage[cmex10]{amsmath}
\usepackage{mathtools}
\usepackage{amsmath}
\usepackage{commath}
%\usepackage{amsthm}https://www.overleaf.com/project/657d6a281358e1782949a240
\interdisplaylinepenalty=2500
%\savesymbol{iint}
\usepackage{txfonts}
%\restoresymbol{TXF}{iint}
\usepackage{wasysym}
\usepackage{amsthm}
\usepackage{mathrsfs}
\usepackage{txfonts}
\let\vec\mathbf{}
\usepackage{stfloats}
\usepackage{float}
\usepackage{cite}
\usepackage{cases}
\usepackage{subfig}
%\usepackage{xtab}
\usepackage{longtable}
\usepackage{multirow}
%\usepackage{algorithm}
\usepackage{amssymb}
%\usepackage{algpseudocode}
\usepackage{enumitem}
\usepackage{mathtools}
%\usepackage{eenrc}
%\usepackage[framemethod=tikz]{mdframed}
\usepackage{listings}
%\usepackage{listings}
\usepackage[latin1]{inputenc}
%%\usepackage{color}{   
%%\usepackage{lscape}
\usepackage{textcomp}
\usepackage{titling}
\usepackage{hyperref}
%\usepackage{fulbigskip}   
\usepackage{tikz}
\usepackage{graphicx}
\lstset{
  frame=single,
  breaklines=true
}
\let\vec\mathbf{}
\usepackage{enumitem}
\usepackage{graphicx}
\usepackage{siunitx}
\let\vec\mathbf{}
\usepackage{enumitem}
\usepackage{graphicx}
\usepackage{enumitem}
\usepackage{tfrupee}
\usepackage{amsmath}
\usepackage{amssymb}
\usepackage{mwe} % for blindtext and example-image-a in example
\usepackage{wrapfig}
\graphicspath{{figs/}}
\providecommand{\cbrak}[1]{\ensuremath{\left\{#1\right\}}}
\providecommand{\brak}[1]{\ensuremath{\left(#1\right)}}
%\providecommand{\norm}[1]{\left\lVert#1\right\rVert}
\newcommand{\myvec}[1]{\ensuremath{\begin{pmatrix}#1\end{pmatrix}}}
\newcommand{\augvec}[3]{\ensuremath{\begin{amatrix}{#1|#2}#3\end{amatrix}}}
\newcommand{\mydet}[1]{\ensuremath{\begin{vmatrix}#1\end{vmatrix}}}
\usepackage{subfig}\graphicspath{{/storage/self/primary/Download/latexnew/fig}}

%\newcommand{\abs}[1]{\lvert#1\rvert}
%\newcommand{\norm}[1]{\lVert#1\rVert}
\providecommand{\sbrak}[1]{\ensuremath{{}\left[#1\right]}}
\providecommand{\brak}[1]{\ensuremath{\left(#1\right)}}
\providecommand{\cbrak}[1]{\ensuremath{\left\{#1\right\}}}
%\newcommand{\myvec}[1]{\ensuremath{\begin{pmatrix}#1%\end{pmatrix}}}
\newcommand{\myaugvec}[2]{\ensuremath{\begin{amatrix}{#1}#2\end{amatrix}}}
%\newcommand{\mydet}[1]{\ensuremath{\begin{vmatrix}#1%\end{vmatrix}}}

\begin{document}
\title{\textbf{MATH-COMPUTING}}
\maketitle
\begin{enumerate}
 
    \item \textbf{Question(MATH-12.10.5.17):}
       Let $\vec{a}$ and $\vec{b}$ be two unit vectors and $\theta$ is the angle between them. Then $\vec{a}+\vec{b}$ is a unit vector.
    
 \begin{enumerate}[label=(\Alph*)]                     
 \item $\theta$=$\frac{\pi}{4}$
 \item $\theta$=$\frac{\pi}{3}$
  \item $\theta$=$\frac{\pi}{2}$
   \item $\theta$=$\frac{2\pi}{3}$
   \end{enumerate}

 \textbf{solution:}

Given,
\begin{align}
	\norm{\vec{a}} = \norm{\vec{b}}=1 
 \label{eq:eq1}
  \end{align}
 \begin{align}
	\norm{\vec{a}+\vec{b}}=1
 \label{eq:eq0}
 \end{align}
  and, its magnitude will be as:
 \begin{align}
magnitude=\sqrt{\vec{a}^2+\vec{b}^2}
 \end{align}
\begin{align}
magnitude =\sqrt{1^2+1^2}=\sqrt{2}
 \end{align}
Squaring on both sides of \eqref{eq:eq0}, we get
\begin{align}
	\norm{\vec{a}+\vec{b}}^2=1^2
\\	
	\implies \norm{\vec{a}}^2 + \norm{\vec{b}}^2 + 2\vec{a}^{\top}\vec{b} = 1
 \label{eq:eq2}
\end{align}

Substituting \eqref{eq:eq2} in \eqref{eq:eq1}, we get

\begin{align}
	\implies 1+1+2(\norm{\vec{a}}\norm{\vec{b}}\cos{\theta})=1
	\\
	\implies 2+2(\norm{\vec{a}}\norm{\vec{b}}\cos{\theta})=1
        \\
	\implies 2(\norm{\vec{a}}\norm{\vec{b}}\cos{\theta})=-1
	\\
	\implies (\norm{\vec{a}}\norm{\vec{b}}\cos{\theta})=\frac{-1}{2}
 \label{eq:eq3}
\end{align}

Substituting \eqref{eq:eq1} in \eqref{eq:eq3}, we get
\begin{align}
	\implies \cos{\theta}=\frac{-1}{2}
	\\
	\implies \theta=\frac{2\pi}{3}
\end{align}

The point f "a" is:
\begin{align}
    \norm{\vec{a}}=\myvec{a_1 \\ a_2} 
    \end{align}
    The equation of the magnitude of a 2-dimensional vector is: 
    \begin{align}
    a_1^2 +a_2^2 =1
    \end{align}
    Assume value of a1 is given and let a1=a1 
  \begin{align}
   \implies   a_2^2 = 1-a_1^2 
    \end{align}
    \begin{align}
 \implies     a_2 = \sqrt{1-a_1^2}
 \label{eq:eq10}
     \end{align}
And, 
The point of "b" is:
\begin{align}
    \norm{\vec{b}}=\myvec{b_1 \\ b_2} 
    \end{align}
    The equation of the magnitude of a 2-dimensional vector is: 
    \begin{align}
    b_1^2 +b_2^2 =1
    \end{align}
  \begin{align}
   \implies   b_2^2 = 1-b_1^2 
    \end{align}
    \begin{align}
 \implies     b_2 = \sqrt{1-b_1^2}
 \label{eq:eq6}
     \end{align}
Since, 
\begin{align}
 {a_1}{b_1} + {a_2}{b_2} = \frac{-1}{2}
\end{align}
\begin{align}
\implies {a_1}{b_1} + \sqrt{1-{a_1}^2} \sqrt{1-{b_1}^2}  = \frac{-1}{2}
\end{align}
\begin{align}
\implies  \sqrt{{1-{b_1}^2-{a_1}^2+{a_1}^2{b_1}^2}} =  \frac{-1}{2}- {a_1}{b_1} 
\end{align}
Squaring on both the sides:
\begin{align}
\implies  {1-{b_1}^2-{a_1}^2+{a_1}^2{b_1}^2} =  \brak{\frac{-1}{2}- {a_1}{b_1}}^2
    \end{align}
    \begin{align}
\implies {1-{b_1}^2-{a_1}^2+{a_1}^2{b_1}^2} = \brak{\frac{-1}{2}}^2 + (a_1b_1)^2 - 2\brak{\frac{-1}{2}}(a_1b_1)
    \end{align}
    \begin{align}
 \implies   {1-{b_1}^2-{a_1}^2+{a_1}^2{b_1}^2} = \frac{1}{4} +(a_1b_1)^2+a_1b_1
    \end{align}
    \begin{align}
   \implies -{b_1}^2-{a_1}^2+{a_1b_1} = \frac{1}{4}-1
    \end{align}
  \begin{align}
  \implies {b_1}^2+{a_1}^2+{a_1b_1}=\frac{3}{4}
  \end{align}
  \begin{align}
 \implies {b_1}^2+{a_1b_1}+\brak{{a_1}^2-\frac{3}{4}}=0
  \end{align}
  We know the formula of Quadratic equation to find roots: 
  \begin{align}
   X = \frac{-B \pm \sqrt{B^2 - 4AC}}{2A}
\end{align} 
\begin{align}
\implies  b_1 = \frac{(-a_1) \pm \sqrt{a_1^2 - 4(a_1^2-\frac{3}{4})}}{2}
  \end{align}  
  \begin{align}
\implies  b_1 = \frac{ (-a_1)\pm \sqrt{-3a_1^2+3}}{2}
\label{eq:eq5}
  \end{align}
  Substituting \eqref{eq:eq5} in \eqref{eq:eq6}, we get
  \begin{align}
b_2=\sqrt{1-b_1^2}
  \end{align}
  \begin{align}
    \implies b_2=\sqrt{1-\brak{\frac{-a_1 \pm \sqrt{3-a_1^2}}{2}}}
    \end{align}
     \begin{align}
 \implies b_2=\sqrt{1-\frac{1}{4}\brak{-a_1 \pm \sqrt{3-a_1^2}}^2}
 \end{align}
 \begin{align}
  \implies b_2= \sqrt{1-\frac{1}{4}\brak{-a_1^2 + 2a_1\sqrt{3-3a_1^2}+3-3a_1^2}}
   \end{align}
   \begin{align}
   \implies  b_2= \sqrt{1-\frac{1}{4}\brak{3-2a_1^2 \pm 2a_1\sqrt{3-3a_1^2}}}
     \end{align}
   \begin{align}
   \implies  b_2= \sqrt{1-\frac{1}{4}\brak{3-2a_1 \brak{2a_1+\sqrt{3-3a_1^2}}}}
   \end{align}
   \begin{align}
   \implies b_2=\sqrt{1-\frac{3}{4}-\frac{a_1}{2}\brak{2a_1 + \sqrt{3-3a_1^2}}}
   \end{align}
     \begin{align}
     \implies b_2= \sqrt{\frac{1}{4}-\frac{a_1}{2}\brak{2a_1+\sqrt{3-3a_1^2}}}
     \label{eq:eq11}
      \end{align}
       Therefore, equations \eqref{eq:eq10},\eqref{eq:eq5} ,\eqref{eq:eq11} gives the values of a2,b1 and b2 
      \begin{align}
 \implies   a_2 = \sqrt{1-a_1^2}
     \end{align}
   \begin{align}
\implies  b_1 = \frac{ (-a_1)\pm \sqrt{-3a_1^2+3}}{2}
  \end{align}  
  \begin{align}
   \implies b_2= \sqrt{\frac{1}{4}-\frac{a_1}{2}\brak{2a_1+\sqrt{3-3a_1^2}}}
      \end{align}
      The angle of the two unit vectors will be:
      \begin{align}
         \theta = \cos^{-1}\brak{{\frac{\vec{a}\vec{b}}{\mydet{\vec{a}}\mydet{\vec{b}}}}}
      \end{align}
      Where,
       \begin{align}
      \vec{ab}=a_1b_1 +a_2b_2
       \end{align}
        \begin{align}
      \mydet{\vec{a}} = \sqrt{a_1^2+a_2^2}
       \end{align}
        \begin{align}
      \mydet{\vec{b}} = \sqrt{b_1^2+b_2^2}
       \end{align}
       Let's assume the input value a1 =1, then:
       \begin{align}
    a_2=\sqrt{1-a_1} = \sqrt{1-1} =0
       \end{align}
        \begin{align}
b_1=  \frac{ (-a_1)\pm \sqrt{-3a_1^2+3}}{2} =  \frac{ (-1)\pm \sqrt{-3(1)^2+3}}{2} = \frac{-1}{2}
          \end{align}
          \begin{align}
   b_2= \sqrt{\frac{1}{4}-\frac{a_1}{2}\brak{2a_1+\sqrt{3-3a_1^2}}} = \sqrt{\frac{1}{4}-\frac{1}{2}\brak{2(1)+\sqrt{3-3(1)^2}}} =\sqrt{\frac{-3}{4}}
       \end{align}
  \end{enumerate}
\end{document}o
